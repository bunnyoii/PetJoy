\section{数据库设计概述}\label{sec:Overview}

在本节(\cref{sec:Overview})中,我们将对数据库设计进行概述。

\subsection{系统数据需求分析}

在\cref{sec:RequirementsAnalysis}中,我们将对系统数据需求进行分析。

在需求分析阶段,我们将详细讨论如何收集和分析用户需求以及业务规则。这是设计数据库前的首要步骤,目的是确定所需数据的类型、范围及其相互关系。这一阶段的输出是一个详尽的需求说明书,为后续的概念设计奠定基础。

\subsection{数据库概念设计}

在\cref{sec:ConceptualDesign}中,我们将介绍数据库实体及相关属性、数据库总体E-R图和数据库模块E-R图。

概念设计阶段关注于根据需求分析结果建立数据库的高层结构。这一阶段的主要任务是定义数据模型,通常使用实体-关系模型(E-R图)来表示。我们将详细描述数据库中的实体、属性、关系及其约束,形成一个完整的E-R图,覆盖数据库的整体结构及各个模块。

\subsection{数据库逻辑设计}

在\cref{sec:LogicalDesign}中,我们将介绍数据库逻辑设计,包括数据库关系模式图、表设计、数据库设计、数据定义语言(DDL)设计和触发器设计。

在逻辑设计阶段,将概念设计转化为逻辑模型。这一阶段主要包括定义关系模式、正规化过程以避免数据冗余、以及设计逻辑表结构。逻辑设计的结果是一系列详细的关系模式图和数据表设计,以及相应的数据定义语言(DDL)代码,用于实际建立数据库。

在数据库逻辑设计中,触发器设计也是一个关键部分,它用于在特定事件发生时自动执行预定义的操作。触发器可以用于维护数据完整性、实现复杂的业务规则、自动生成审计日志等。我们在插入、更新或删除操作上定义触发器,以确保数据库状态的一致性和可靠性。