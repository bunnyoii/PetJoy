\section{需求分析与逻辑架构设计}\label{sec:Requirements_Analysis_and_Logical_Architecture_Design}

\subsection{需求分析}

在当今社会,宠物已成为家庭生活的重要成员,宠物主对各类相关信息和服务的需求迅速增长,不仅包括基础护理,还涉及社交互动、新闻资讯、宠物领养及百科知识等方面。传统的单一服务模式难以满足现代宠物主的多样化需求,他们期待一个综合性的互动平台,能在同一平台上获取信息、分享经验、进行宠物领养,并提升与宠物的互动体验。

PetJoy 项目旨在应对这一需求,通过整合社区互动、新闻资讯、宠物领养、百科知识及智能AI支持等模块,为宠物主提供一站式解决方案,满足他们在宠物养护和信息获取等方面的需求。平台不仅关注实用功能,还致力于构建一个温馨且充满乐趣的宠物生态系统。通过先进技术和用户反馈机制,PetJoy 持续优化平台功能和用户体验,力求成为宠物主生活中不可或缺的一部分。

未来,PetJoy 团队将进一步扩展平台功能,增强社交互动和社区支持,并整合更多相关服务,提升宠物主与宠物之间的生活质量,实现宠物生态系统的可持续发展。

\subsection{逻辑架构设计}

PetJoy 的逻辑架构设计围绕几个关键模块展开,包括宠物信息管理、宠物领养、宠物新闻、社区支持,以及智能宠物AI功能。通过先进的网络技术和友好的用户界面,平台为用户提供了一个简洁易用的宠物社区。

在架构设计上,PetJoy 分为几个层次。首先是数据层,负责存储和管理宠物信息、用户数据、新闻资讯以及AI模型。然后是业务逻辑层,这一层处理宠物信息管理、领养流程、新闻推送和社区互动等核心功能,同时整合智能宠物AI技术,使用户体验更智能、更贴心。最后是用户界面层,它为用户提供了一个直观的操作界面,方便用户轻松访问和使用各种功能。在此我们还考虑到了多种错误处理和安全性保护,旨在让用户交互更加舒适便捷安心。

整个平台采用模块化设计,不仅使功能扩展变得容易,还支持个性化的服务和互动。我们的网站分为十个主要模块,分别是用户及管理员账户管理操作、首页与上下导航栏、宠物社区、宠物新闻、宠物领养、宠物百科、宠物AI、搜索框和异常处理。每个模块都是一个子系统,具有自己的功能和页面。这些子系统之间也有一定的联系和依赖关系。

\begin{itemize}
	\item \textbf{账号管理子系统}

	该子系统是其他功能模块的基础,主要负责账号的登录、注册、登出,以及账号的安全性保护、隐私设置和个人信息的展示与管理。确保账号处理的正常和数据的完整性,是系统稳定运行和避免崩溃的关键。

	\item \textbf{管理员子系统}

	该子系统为项目管理员提供了一系列管理工具,包括封禁用户账号、管理帖子、发布与管理新闻等功能,以确保平台内容的合规性和社区的健康发展。

	\item \textbf{首页子系统}

	该子系统主要负责首页的跳转操作,涵盖上方导航栏的跳转、天气显示、亮暗模式调整、个人主页访问、语言设置,以及下方导航栏中的隐私政策、意见反馈、关于宠悦等链接,旨在提升用户的整体交互体验。

	\item \textbf{宠物社区子系统}

	该子系统提供了丰富的宠物社区互动功能,包括帖子分类、用户端的点赞、评论和举报功能,以及管理员端的帖子管理功能,如置顶、封禁等,确保社区环境的活跃与秩序。

	\item \textbf{宠物新闻子系统}

	该子系统专注于宠物新闻的管理和互动,支持新闻分类、用户端的点赞、评论和举报功能,以及管理员端的新闻管理功能,如置顶和删除,保障新闻内容的及时性和正确性。

	\item \textbf{宠物领养子系统}

	该子系统围绕宠物领养提供了一系列功能,包括领养信息的查阅和下载、宠物分类浏览以及发布领养信息,帮助用户快速找到合适的领养对象并了解相关信息。

	\item \textbf{宠物百科子系统}

	该子系统专为宠物知识的展示与查询设计,提供交互性强的菜单导航、随机宠物推荐展示,以及详细的宠物分类和信息展示,包括基本信息、起源地图和护理指导等,帮助用户全面了解各种宠物。

	\item \textbf{宠物AI子系统}

	该子系统提供了与宠物AI互动的功能,用户可以通过示例问题进行交互,并在过程中向AI提问,获取智能化的答案和建议。

	\item \textbf{搜索框子系统}

	该子系统支持全站搜索功能,在不同页面提供不同形式和功能的搜索框,满足用户的多样化搜索需求。

	\item \textbf{异常处理子系统}

	该子系统负责在异常事件发生时的处理,如显示404页面以应对地址错误,或在数据获取失败时提供相应的提示信息,确保用户能够获得明确的反馈。
	
\end{itemize}