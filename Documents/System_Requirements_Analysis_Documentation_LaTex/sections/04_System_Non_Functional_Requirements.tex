\section{系统非功能性需求}\label{sec:System_Non_Functional_Requirements}

\subsection{安全性需求}

数据安全性、数据持久化和隐私保护是现代应用系统中的关键要求。用户的敏感信息,如个人身份数据、交易记录等,必须得到有效保护,以防止数据泄露、篡改或丢失。这不仅关系到用户的隐私权,还影响到系统的信誉和法律合规性。确保数据的安全性和隐私保护能够增强用户信任,并避免因数据泄露而导致的潜在法律责任和经济损失。

\subsubsection{数据安全}

数据安全是保护用户数据免受未经授权访问、篡改和破坏的重要措施。通过加密、认证和访问控制等技术手段,确保用户数据的机密性、完整性和可用性,防止数据泄露和滥用。数据安全性不仅是用户信任的基础,也是系统稳定运行和合规经营的前提。

\begin{itemize}
	\item \textbf{密码哈希处理}:使用 \texttt{SHA-256} 加密算法对用户密码进行哈希处理,防止明文存储密码,从而提高密码存储的安全性。SHA-256 是一种不可逆的哈希算法,能有效避免密码被破解。
	\item \textbf{JWT Token 认证}:通过使用 \texttt{JSON Web Token (JWT)} 进行后端用户身份验证,确保只有经过授权的用户才能访问受保护的资源。
	\item \textbf{SQL 防注入}:采用 \texttt{ORM (Object-Relational Mapping)} 组件防止 SQL 注入攻击,确保数据库操作的安全性。
	\item \textbf{防火墙和端口控制}:严格控制服务器端口访问,仅开放必要的服务端口,如 HTTP、HTTPS、SSH 等,以防止未经授权的访问。
	\item \textbf{用户权限管理}:通过细化的权限管理,确保系统中不同角色的用户只能访问其权限范围内的数据和功能。
\end{itemize}

\subsubsection{数据持久化}

数据持久化是确保数据在系统重启或故障时不会丢失的重要机制。通过数据备份、容灾和故障恢复等手段,确保数据的完整性和可用性,提高系统的稳定性和可靠性。数据持久化不仅是系统正常运行的基础,也是保障用户数据安全和隐私的重要手段。

\begin{itemize}
	\item \textbf{Docker 数据持久化}:在 Docker 容器中配置数据持久化机制,确保数据库和关键数据在容器重启或故障恢复后仍能保持完整。
	\item \textbf{数据库备份}:定期对数据库进行备份,以保证在数据丢失或损坏时能够快速恢复。
\end{itemize}

\subsubsection{隐私保护}

隐私保护是保护用户个人信息和隐私权的重要措施。通过数据加密、匿名化和访问控制等技术手段,确保用户的个人信息不被未经授权的访问和使用,避免用户信息泄露和滥用。隐私保护不仅是用户权益的保障,也是系统合规运营和社会责任的体现。

\begin{itemize}
	\item \textbf{数据加密}:即使在数据传输过程中也将使用 \texttt{HTTPS} 加密,尽管当前未购买域名,但计划使用免费 \texttt{SSL} 证书(如 \texttt{Let's Encrypt})为 HTTP 连接启用 HTTPS 加密。
	\item \textbf{DDoS 防护}:使用 \texttt{阿里云} 的 DDoS 防护服务,通过流量过滤、监控和响应机制来防止恶意流量攻击。
\end{itemize}

\subsection{性能需求}

性能优化是确保数据库和系统高效运行的关键。优化数据库性能可以显著提高系统的响应速度、处理能力以及用户体验。性能优化还包括资源的合理利用和维护系统的高可用性,这对提升用户满意度和系统稳定性至关重要。

\subsubsection{性能优化方法}

\begin{itemize}
	\item \textbf{索引创建}:为高频查询的字段(如主键、外键及常用查询条件字段)创建索引,以加快数据检索速度。
	\item \textbf{查询优化}:编写高效的 SQL 查询语句,减少不必要的查询开销,提升数据库性能。
	\item \textbf{数据库设计优化}:进行合理的数据库结构设计,减少冗余数据,提高数据处理效率。
	\item \textbf{缓存机制}:通过缓存机制减少数据库直接访问,提升系统整体性能。
	\item \textbf{定期维护和监控}:对数据库进行定期维护和监控,及时发现并解决潜在性能问题。
\end{itemize}

以下是经过修改后的内容,使其更加专业完备:

\subsection{用户交互需求}

\subsubsection{显示需求}

\begin{itemize}
	\item \textbf{亮暗模式支持}:提供亮模式和暗模式的主题切换功能,允许用户根据个人偏好或环境光线条件选择适合的显示模式。采用柔和的颜色方案,以提升界面的可读性和视觉舒适度。
	\item \textbf{多语言国际化支持}:用户界面文本(包括菜单、按钮、提示信息、错误消息等)支持多达十种语言的翻译(包括汉语、英语、法语、德语、西班牙语、葡萄牙语、俄语、意大利语、日语和韩语)。用户可以根据需求选择语言,系统将自动切换界面内容。同时支持动态语言切换功能,用户无需重新登录或刷新页面即可随时切换语言。
	\item \textbf{响应式设计与设备兼容性}:用户界面采用响应式设计,能够自适应不同设备的屏幕尺寸(如桌面、平板、手机等)。布局将根据屏幕宽度自动调整,确保内容在各种设备上均能正常显示,并提供一致的用户体验。
	\item \textbf{加载状态与进度反馈机制}:在数据加载、提交或处理过程中,界面将显示加载指示器,以告知用户当前操作的进度,避免因等待时间过长引发的混淆。例如,在天气界面和宠物AI界面设置了加载过渡提示,顶部导航栏还设有页面加载滚动条以提供视觉反馈。
	\item \textbf{错误与提示信息展示}:当发生错误或操作失败时,系统将提供清晰且用户友好的错误信息,说明问题原因并提供解决建议(如网络问题时建议刷新界面)。在用户执行关键操作之前(如提交举报或删除帖子/新闻),系统将显示确认对话框,以确保用户操作的准确性,并反馈成功或失败的结果信息。
\end{itemize}

\subsubsection{互动需求}

\begin{itemize}
	\item \textbf{输入验证与实时反馈}:在用户输入数据时,系统将提供实时的输入验证,并即时提示错误或不符合要求的输入(如格式错误、必填项为空等)。例如,在登录注册界面中进行手机号验证,并限制输入框的最大字符数,以确保数据的有效性和完整性。
	\item \textbf{操作提醒与确认机制}:在用户执行关键操作(如删除、提交举报等)前,系统将显示确认提醒,以防止用户误操作。此功能在举报提交和帖子/新闻删除界面尤为重要,确保用户的操作意图明确。
	\item \textbf{帮助与反馈机制整合}:系统提供新手引导功能,帮助用户快速理解系统功能和操作步骤。同时,在页面底部设置反馈机制,用户可以直接提交问题、建议或评价,增强用户与开发团队之间的互动与沟通。
	\item \textbf{导航与界面结构优化}:提供清晰的导航结构,使用户能够轻松在不同层级之间切换,避免在复杂界面中迷失。例如,在宠物领养和宠物百科等界面设置了面包屑导航,显示用户在系统中的路径位置,允许用户快速返回上一级或主页。
	\item \textbf{动态内容与实时更新机制}:系统支持多界面(如首页、关于宠悦界面)的动态内容展示,通过设置图片内容浮动效果增强界面的互动性。此外,在界面元素切换或内容更新时,应用适当的动画与过渡效果,提升用户体验。在宠物百科界面中,支持宠物推荐内容的刷新,进一步增强用户体验的趣味性与互动性。
\end{itemize}