\section{数据需求}\label{sec:Data_Requirements}

\subsection{用户表相关数据需求}

\subsubsection{用户表}

用户表是管理用户账户的核心数据结构,负责身份识别和账户管理。它包含了用户的基本信息如唯一标识符、用户名、密码哈希等,并通过加密哈希技术保护密码安全。用户表还记录了联系方式、注册日期、上次登录日期、用户角色和状态,用于多因素验证、账户管理和权限控制。此外,用户表存储了个人资料信息和用户互动数据,这些信息支持平台的社交功能,并可用于用户奖励机制和推荐系统,增强用户的参与度和活跃度。

\subsubsection{用户设置表}

用户设置表通过个性化界面和隐私控制,增强了用户对账户体验的控制权。核心功能包括细致的隐私设置,允许用户决定哪些个人信息和社交数据公开或保密。此外,用户可以管理通知接收偏好,减少干扰,并通过互动设置控制社交关系。这些功能共同提升了平台的个性化和隐私保护,使用户享有更安全、舒适的使用体验。

\subsubsection{用户打卡表}

用户打卡表通过记录用户每日登录活动,旨在提升用户的参与度和粘性。每次登录时,系统会创建一条包含打卡记录ID、用户ID和打卡时间的记录,这些数据用于分析用户活跃度和使用习惯,帮助平台优化功能、推荐内容,并制定奖励机制以鼓励连续登录。此外,打卡数据还支持用户行为分析,提升服务质量,并有助于检测和防范异常行为,确保平台的公平性和安全性。

\subsubsection{用户关注表}

用户关注表记录并管理用户之间的关注关系,旨在提升社交互动和平台粘性。每次关注或取消关注时,系统会更新记录,追踪用户的社交网络发展。这些数据帮助平台优化内容推荐,提升用户参与感,并通过激励机制促进更多的社交互动。关注数据还支持用户行为分析,精确市场细分,识别异常行为,确保平台的安全性和公平性。

\subsubsection{用户留言表}

用户留言表提供了一个平台内的直接消息交换方式,促进用户之间的沟通和社区互动,增强合作氛围。每条留言记录包含唯一的标识符、用户信息、内容和时间等,确保消息的可追溯性和数据分析的可能性。留言表鼓励用户互动,支持私人交流和社区讨论,推动社区的活跃性和多样性。此外,留言数据还可用于社区管理和内容审核,帮助平台优化用户体验和维护社区秩序。

\subsubsection{用户反馈表}

用户反馈表为用户提供了提交意见和建议的渠道,帮助平台识别问题、了解用户需求,并推动持续改进。反馈表包含反馈ID、类型、内容、时间、联系方式等关键字段,有助于分类处理反馈并与用户沟通。通过分析反馈数据,平台可以发现共性问题,优化功能,提升用户体验和满意度,从而增强用户的信任和忠诚度,实现平台与用户的双赢。

\subsection{帖子表相关数据需求}

\subsubsection{帖子评论表}

帖子评论表为用户提供了一个互动平台,增强了用户对帖子的参与度和可视性。通过记录评论ID、帖子ID、用户ID、内容、时间、点赞数、点踩数等信息,平台能够追踪互动频率、优化资源分配,并识别用户活跃时间段。点赞数和点踩数反映评论质量,支持内容推荐优化。此外,嵌套评论功能增强了讨论深度和用户互动,有助于社区发展。评论数据的分析还能帮助平台了解用户兴趣,优化内容生成和推荐策略,提升用户满意度。

\subsubsection{帖子表与帖子分类表}

帖子评论表为用户提供了一个互动平台,增强了用户对帖子的参与度和可视性。通过记录评论ID、帖子ID、用户ID、内容、时间、点赞数、点踩数等信息,平台能够追踪互动频率、优化资源分配,并识别用户活跃时间段。点赞数和点踩数反映评论质量,支持内容推荐优化。此外,嵌套评论功能增强了讨论深度和用户互动,有助于社区发展。评论数据的分析还能帮助平台了解用户兴趣,优化内容生成和推荐策略,提升用户满意度。

\subsubsection{帖子(评论)点踩表}

点赞和点踩表记录用户对帖子和评论的反馈,提供了一种简单的互动方式,帮助平台和内容创作者了解内容受欢迎程度。每条记录包含用户ID、帖子或评论ID及反馈类型,平台通过这些数据分析用户喜好,优化内容推荐算法。创作者通过点赞和点踩数量调整创作策略,平台运营者则利用这些数据提升内容质量和用户体验,并进行个性化推荐,增强用户满意度和平台活跃度。

\subsubsection{帖子收藏表}

帖子收藏表为用户提供了便捷的方式来保存感兴趣的内容,提升再访问率和用户体验。通过分析收藏数据,平台可以了解用户兴趣,优化内容推荐,并提供个性化的内容建议。收藏功能还帮助内容创作者调整创作策略,提升内容质量和吸引力。此外,平台可以利用收藏数据优化广告投放策略,提高广告效果和收益,增强用户的参与度和满意度。

\subsubsection{帖子(评论)举报表}

帖子和帖子评论举报表为用户提供了标记不当内容的机制,确保社区的健康和安全。通过记录举报详情,包括举报ID、帖子或评论ID、用户ID、举报理由和时间,平台可以系统化管理和处理举报事件,维护内容质量。举报数据帮助运营团队快速响应违规行为,优化管理策略,并预防问题内容的出现。这种机制不仅保护用户免受有害内容的影响,还增强了用户对平台的信任和参与度,共同维护社区的和谐环境。

\subsection{新闻表相关数据需求}

\subsubsection{新闻表和新闻标签表}

新闻表和新闻标签表共同构建了平台的新闻发布与管理系统,提升了内容的可发现性和用户参与度。新闻标签表通过唯一的标签ID和标签名称分类新闻,使用户和系统能按主题或兴趣过滤和搜索新闻。新闻表存储了新闻的详细信息,包括标题、内容、作者、标签、时间、封面图片和用户互动数据,如点赞、点踩、收藏和评论数。这些设计确保新闻内容的组织、管理和展示,提高了新闻的可访问性和用户体验,同时通过互动数据优化内容推荐。

\subsubsection{新论评论表}

新闻评论表为用户提供了一个互动平台,使他们能够对新闻内容进行反馈和讨论,增强了内容的互动性和可视性。表中包含评论ID、新闻ID、用户ID、父评论ID、评论内容、时间、点赞数和点踩数等字段,确保评论的唯一性和可追溯性。嵌套评论功能促进了多层次讨论和用户互动。通过分析点赞、点踩和评论数据,平台可以优化内容推荐和管理策略,提升用户体验和平台活跃度。

\subsubsection{新闻(评论)点踩表}

新闻及其评论的点赞和点踩表记录了用户对内容的正面或负面反馈,为评估内容受欢迎程度和优化内容推荐提供了关键数据。表中记录了用户ID、新闻或评论ID、反馈类型和反馈时间,帮助内容创作者和平台运营者评估内容质量和调整策略。点赞和点踩数据不仅用于个性化推荐,提升用户体验,还支持内容审核和管理,确保平台内容的质量和用户的满意度,增强整体活跃度。

\subsubsection{新闻收藏表}

新闻收藏表为用户提供了保存感兴趣新闻的便捷方式,提升内容的再访问率和用户参与度。通过记录用户ID、新闻ID和收藏时间,平台可以分析用户兴趣,优化内容推荐和发布策略,增强用户体验。收藏数据帮助平台识别用户的长期偏好,进行个性化推荐,并优化广告投放,提高广告效果和平台收益,进一步提升用户满意度和平台活跃度。

\subsubsection{新闻(评论)举报表}

新闻评论举报表为用户提供了一种标记不当评论的机制,维护社区的健康和安全。每条举报记录包含唯一的举报ID、举报者和被举报者的ID、相关评论和新闻的ID、举报原因、时间及处理状态。这些信息确保举报内容的可追溯性和处理的系统性。通过分析举报数据,平台可以识别常见违规行为,改进审核机制,提升内容质量和用户信任感。有效的举报处理还鼓励用户积极参与社区管理,促进社区的和谐发展。

\subsection{其他}

\subsubsection{通知表}

通知表是平台的重要组件,负责记录和管理用户的实时通知,确保重要信息及时传递,增强用户参与度。表中包含通知ID、接收者ID、发送者ID、通知类型、内容、时间、是否已读状态,以及相关内容的ID。这些字段帮助平台精确管理通知,提供个性化的用户体验,并确保用户不会错过重要信息。通过分析通知数据,平台可以优化通知机制,提升用户满意度和持续活跃度,同时调整通知策略以更好地满足用户需求。

\subsubsection{宠物分类/子类表}

宠物分类和子类表是宠物百科的核心数据结构,提供详尽的宠物品种信息。宠物分类表定义了高级分类,如狗、猫等,帮助用户快速找到感兴趣的种类,每个分类包含唯一ID、名称、描述和图片链接。宠物子类表进一步详细描述了具体品种的信息,包括起源、体型、毛色、寿命、性格和饮食习惯。这些信息帮助用户更好地选择和护理宠物,增强宠物知识。平台通过这些系统化的分类和丰富的子类信息,提升用户体验、专业性和可信度。

\subsubsection{宠物护理表}

宠物护理指导表提供了详细的宠物护理建议,包括日常护理、健康监测、饲养技巧等内容,帮助宠物主人提升照顾宠物的能力。该表包含唯一的指导ID、相关宠物子类ID、标题和具体内容,确保每条建议的针对性和完整性。通过提供清晰的护理步骤,平台提升了宠物的生活质量和用户对平台的信任。平台还可以根据用户的使用情况优化内容推荐,定期更新护理信息,增强平台的黏性和用户满意度。

\subsubsection{宠物领养表}

宠物领养表是宠物领养服务的核心数据支持系统,记录了宠物的名称、年龄、品种、健康状况、位置、领养状态等详细信息,帮助潜在领养者全面了解并选择合适的宠物。表中的健康状况、疫苗接种记录和特殊护理需求等信息确保了领养过程的透明度,提高了领养成功率。通过分析这些数据,平台可以优化个性化领养推荐和改进领养流程,提升用户体验和整体服务质量,帮助宠物找到合适的家庭。

\subsubsection{开发团队表}

开发团队表记录了平台开发团队成员的详细信息,如姓名、学校、电子邮箱、GitHub 用户名和个人资料链接。该表旨在提升平台的透明度和可信度,让用户了解负责开发和维护服务的团队成员,增强用户对平台的信任感。