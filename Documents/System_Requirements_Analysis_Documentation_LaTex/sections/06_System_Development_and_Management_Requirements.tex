\section{系统开发与管理需求}\label{sec:System_Development_and_Management_Requirements}

\subsection{系统环境配置}

\subsubsection{硬件需求}
在服务器方面,需要配备多核心处理器、充足的RAM以及足够的存储空间以支持应用及其数据的存储需求,并且需具备高带宽网络连接以支持高速数据传输;此外,服务器的有效期应在项目生命周期内。

客户端则应配置高性能处理器、大容量内存和存储以确保良好的用户体验。

\subsubsection{软件需求}
从软件角度看,服务器端推荐使用基于Linux的长期支持版操作系统,而客户端则应采用现代版本的Windows操作系统。

中间件包括Web服务器和应用服务器,并且系统支持容器化的数据库服务,例如Oracle或其他兼容产品,要求具有数据持久化能力和定期备份机制到远程存储位置;数据库连接可通过专业的管理工具进行,例如JetBrains DataGrip,并且需要安全配置,包括启用SSL/TLS加密和严格的访问控制。

\subsubsection{网络环境需求}
在网络环境中,内部网络应支持高速、低延迟的服务器间通信,外部网络需支持公共互联网访问以确保客户端可以远程访问服务器;同时,需配置防火墙保护服务器免受未经授权的访问,并使用SSL/TLS加密保证数据传输安全;为了应对高并发访问,系统应支持负载均衡策略;最后,还需要实施网络监控工具以及时发现网络问题和异常。

\subsection{项目进度安排}

PetJoy 项目需要一个详细的进度安排,以明确目标和任务、提高团队效率、管理潜在风险、优化资源配置、跟踪项目进展,并促进团队协作与沟通。进度安排不仅是团队管理的重要工具,还展示了项目的计划性和可行性,确保项目能够按时、高质量地完成并成功上线。

\subsubsection{第一阶段:前期设计}

在前期设计阶段,我们将专注于用户调研,以确保 PetJoy 平台能够准确满足目标用户的需求。通过与宠物爱好者进行深入访谈和问卷调查,我们将明确项目的核心用户群体,并识别平台需要解决的关键问题,如宠物爱好者之间的交流障碍。基于调研结果,我们将确定 PetJoy 的主题、图标设计和主色调,以建立统一的品牌形象和视觉吸引力。

\subsubsection{第二阶段:初期开发}

在初期开发阶段,我们将搭建 PetJoy 的技术基础设施。为确保平台的数据安全和高效管理,我们将采用 Oracle 数据库,并部署 OSS(对象存储服务)预留空间,用于存储平台所需的各种资源。前后端开发将采用 Vue3.js、C\# 和 TypeScript 等技术栈,并使用 WebStorm 和 Rider 作为开发环境,以提升开发效率,确保平台的响应速度和用户体验。

\subsubsection{第三阶段:设计迭代}

在产品原型完成后,项目将进入设计迭代阶段。通过内部测试和反馈循环,我们将持续优化平台的功能和用户界面。此阶段将引入新特性,包括整合大语言模型和天气系统等 API,以提升用户互动体验。前端页面将持续进行视觉和功能的优化,而后端将不断优化数据库结构和接口响应速度,确保平台实时满足用户需求。

\subsubsection{第四阶段:产品收尾}

在产品收尾阶段,PetJoy 平台将进行最终的部署和全面上线。此阶段还包括系统维护,确保平台上线后的稳定运行。我们将撰写详细的项目最终报告,记录开发过程和关键决策,为未来的维护、更新及后续项目开发提供参考。

通过这四个阶段的有序推进,PetJoy 平台将逐步完善,最终实现为全球宠物爱好者提供一个高效、智能的宠物管理和社交互动平台的目标。

\subsection{人员培训}

为了确保 PetJoy 平台的成功开发与维护,团队将实施一套全面的人员培训计划,着重培养关键技术技能和最佳实践。该计划涵盖以下领域:

\subsubsection{版本控制(Git)}

团队将采用 Git 进行版本控制,并实施多分支策略,以支持团队并行处理项目的不同部分,从而提高开发效率并减少代码冲突的可能性。团队成员将接受关于有效分支策略的培训,包括创建、合并和管理分支的最佳实践,以确保顺畅协作和无缝集成。

\subsubsection{代码规范}

保持代码的一致性和高质量是团队的首要任务。培训计划将详细讲解项目的编码规范,包括严格的代码审查流程,以确保代码在可读性、可维护性和稳定性方面达到高标准。

\subsubsection{应用程序接口文档}

为简化前端开发并优化后端资源的利用,团队将开发专门的后端 API 文档网站“PetJoy API”。该文档网站将基于 Swagger 开发,以提供交互式、自动化生成的 API 文档。Swagger 允许前端开发人员快速理解和调用后端接口,减少沟通成本并加速开发进程。

培训内容将涵盖如何有效使用这些文档,包括如何通过 Swagger 界面浏览 API 端点、测试 API 请求、查看响应示例及错误代码等。此外,培训还将指导如何报告文档中的潜在问题或接口不一致情况,以确保前后端的无缝协作。通过充分利用该文档网站,团队成员能够更好地协调工作,提高开发效率。

\subsubsection{调试方法}

团队成员将接受关于调试方法的培训,重点介绍本地开发环境的设置和使用。内容包括如何在 localhost 服务器上调试代码,以及利用 Vue3 的响应式框架进行高效的个人进度跟踪和问题排查。

通过这些培训,团队将具备开发、维护 PetJoy 平台所需的技术能力,从而推动项目的持续成功。

\subsection{资源配置}

\subsubsection{服务器资源配置}

为了确保 PetJoy 平台的稳定性和高可用性,本项目选择了阿里云轻量应用服务器,并使用 Docker 进行环境管理和服务部署。此配置旨在实现资源的高效利用和管理的简化。

服务器配置:
\begin{itemize}
    \item \textbf{实例ID:} dfc3f736506748c3ba9ced56b94a81d7
    \item \textbf{实例名称:} 宠悦轻量应用服务器
    \item \textbf{地域:} 中国香港
    \item \textbf{配置信息:} 2 核 vCPU, 4GB 内存, 3072GB 每月流量, 80GB ESSD, 30Mbps 限峰值带宽
    \item \textbf{镜像信息:} Ubuntu 22.04
    \item \textbf{私有 IP 地址:} 172.19.54.72
    \item \textbf{公有 IP 地址:} 47.238.195.140
    \item \textbf{到期时间:} 2024 年 10 月 8 日 00:00:00
    \item \textbf{配置费用:} ¥102.00(优惠金额¥300.00)
\end{itemize}

\subsubsection{数据库资源配置}

为了确保 PetJoy 项目的数据管理高效可靠,我们采用 Oracle 12c 数据库并通过 Docker 实现数据持久化,同时利用阿里云 OSS(对象存储服务)来存储平台所需的数据和资源。

数据库配置:
\begin{itemize}
  \item \textbf{类型}:Oracle 12c
  \item \textbf{持久化方式}:将数据库文件挂载至本地存储路径,以确保数据的持久性。
  \item \textbf{连接方式}:通过轻量应用服务器或 JetBrains DataGrip 进行数据库连接,以支持开发和运维的高效性。
\end{itemize}

\subsubsection{OSS 对象存储预留空间配置}

对象存储服务(OSS) 主要用于存储用户上传的图片、平台资源文件及其他相关数据,确保数据的高可用性和安全性。

(OSS)对象存储配置信息:
\begin{itemize}
\item \textbf{存储容量}:需要预留 500 GB 的对象存储空间,以满足平台用户上传的图片、资源文件及其他数据的存储需求。
\item \textbf{使用地区}:存储空间将部署在中国内地区域,以确保数据的存储位置符合当地的法律法规要求。
\item \textbf{计费方式}:按年包月计费,初始配置为1年,费用为¥118.99 元/年,享受新客户优惠价。未来根据需求可灵活扩展存储空间及服务时长。
\item \textbf{访问方式}:数据通过 API 接口进行上传、下载及管理,确保系统的高可用性和数据的安全性。
\item \textbf{服务协议}:应遵守 OSS 对象存储服务条款,用户在确认购买前需阅读并同意相关服务协议。
\end{itemize}

此配置确保平台能够高效、安全地管理和存储用户数据,满足 PetJoy 项目的长期发展需求。

\subsubsection{软件与开发工具}

我们将选用现代化的开发工具和框架,以确保 PetJoy 平台在开发、测试和部署环节中的高效性和一致性。

开发工具:
\begin{itemize}
  \item \textbf{版本控制}:使用 Git 进行代码版本管理,确保团队协作中的代码变更可追踪、可回溯。
  \item \textbf{集成开发环境(IDE)}:使用 JetBrains 系列工具进行数据库管理和后端接口调试,以提升开发效率。
  \item \textbf{容器化工具}:通过 Docker 管理开发环境和部署流程,简化环境配置,确保一致性和可移植性。
\end{itemize}

技术栈:
\begin{itemize}
  \item \textbf{前端}:采用 Vue3.js 作为主要框架,结合 Element UI 进行用户界面开发,保证用户界面的美观与响应速度。
  \item \textbf{后端}:基于 C\# 进行接口开发,支持高效的 API 服务和数据处理。
\end{itemize}

\subsubsection{安全配置}

通过实施一系列安全措施,保护 PetJoy 平台免受各种安全威胁和网络攻击,确保平台的安全性和稳定性。

\begin{itemize}
    \item \textbf{JWT Token 后端登录鉴权}:使用 JWT(JSON Web Token)进行用户身份验证,确保只有经过授权的用户才能访问受保护的资源。
    \item \textbf{SQL 防注入}:使用 ORM(Object-Relational Mapping)组件来防止 SQL 注入攻击,确保数据库操作的安全性。
    \item \textbf{防火墙端口}:严格控制服务器的端口访问,开放必要的服务端口(如 HTTP、HTTPS、SSH 等),防止未经授权的访问。
    \item \textbf{用户权限设置}:通过细化的用户权限管理,确保系统中不同角色的用户只能访问其权限范围内的数据和功能。
    \item \textbf{数据持久化}:在 Docker 中配置数据持久化机制,确保数据库和关键数据在容器重启或故障恢复后不会丢失。
    \item \textbf{数据库数据定时备份}:定期对数据库进行备份,确保在数据丢失或损坏时能够快速恢复。
    \item \textbf{基于阿里云的云服务器异常检测}:利用阿里云的监控工具实时监控服务器异常,关注 DDoS 攻击等潜在威胁,并及时作出响应。
    \item \textbf{DDoS 防护措施}:
    \begin{itemize}
      \item \textbf{默认端口防护}:避免开放不必要的默认端口,减少潜在攻击面。
      \item \textbf{流量过滤}:通过阿里云的 DDoS 防护服务,对流量进行过滤和清洗,阻止恶意流量。
      \item \textbf{监控与响应}:实时监控网络流量和服务器状态,迅速响应异常流量和攻击行为。
    \end{itemize}
    \item \textbf{HTTPS 加密}:尽管当前未购买域名,但计划使用免费 SSL 证书(如 Let's Encrypt)为 HTTP 连接启用 HTTPS 加密,以增加数据传输的安全性。
\end{itemize}
  
通过以上资源配置和安全措施,PetJoy 项目将确保平台的安全性、稳定性和高效性。团队将持续关注最新的安全威胁和技术发展,及时调整和优化安全策略,保障用户数据和系统的安全。

\subsection{质量保证}
    
在 PetJoy 项目的开发过程中,我们团队始终坚持以高质量为核心目标,确保平台的每一个功能都能够达到用户的期望,并超越他们的需求。为了实现这一目标,我们在用户调研、设计、开发和管理各个层面都制定了严格的质量保证措施。
    
\subsubsection{用户层面的质量保证}
    
我们的质量保证从用户需求的深度挖掘开始。团队在项目初期进行广泛的用户调研,深入分析目标用户的真实痛点和需求。这一过程确保了 PetJoy 的功能设计紧密贴合用户的使用习惯和期望,进一步提升了平台的实用性和用户体验。在产品上线后,我们将通过持续的运维和用户反馈机制,及时调整和优化平台的各项功能,确保用户在使用 PetJoy 时获得高效、舒适的体验。
    
\subsubsection{设计层面的质量保证}
    
在设计阶段,我们实施一套高效且严格的迭代机制。团队通过头脑风暴和批判性思考,确保每一项设计决策都经过深思熟虑和反复推敲。十人团队的集体智慧得以充分发挥,最终使产出的设计方案不仅具备创新性,更经过多次迭代,形成一个功能完善、用户体验优越的产品雏形。
    
\subsubsection{开发层面的质量保证}
    
在开发过程中,PetJoy 团队采用先进的 Git Pipeline 流水线并行开发模式。每一位开发人员的代码都经过严格的代码审查(Code Review),以确保代码的安全性和一致性。此流程不仅为当前开发提供高效的保障,还为后续的维护和升级打下了坚实的基础,使得新加入的开发人员能够迅速上手。此外,我们的开发实践始终以充分利用硬件资源为目标,确保代码不仅仅是实现功能,而是能够高效运行,为用户带来流畅的使用体验。
    
\subsubsection{管理层面的质量保证}
    
在项目管理层面,PetJoy 团队的组长将定期主持小组会议,确保每一位组员都能及时汇报进度,并就开发中遇到的挑战提出解决方案。这种高频次的沟通与反馈机制,使得项目的推进可以始终保持在高效且有条不紊的状态。我们严格遵循项目进度安排,确保每一个开发阶段都能够保质保量地完成。
    
\subsubsection{文档层面的质量保证}
    
为了确保项目的长期可持续性和可维护性,团队会撰写详尽的设计文档、开发文档以及其他相关文档。这些文档不仅为开发人员提供了明确的技术指导,还为管理人员和未来的维护团队提供了全面的参考资料。这一举措大大提升了项目的透明度和可操作性,确保 PetJoy 在未来的发展中依然能够保持高质量的标准。
    
